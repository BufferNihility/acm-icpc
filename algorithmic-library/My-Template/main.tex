\documentclass[titlepage,landscape,a4paper,10pt]{article}
\usepackage{listings, color, fontspec, minted, setspace, titlesec, fancyhdr, dingbat, mdframed, multicol}
\usepackage{graphicx, amssymb, amsmath, textcomp, booktabs}
\usepackage[Chinese]{ucharclasses}
\usepackage[left=1.5cm, right=0.7cm, top=1.7cm, bottom=0.0cm]{geometry}

%configure the top corners
\pagestyle{fancy}
\setlength{\headsep}{0.1cm}
\rhead{Page \thepage}
\lhead{北京交通大学 Beijing JiaoTong University}

%configure space between the two columns
\setlength{\columnsep}{30pt}

%configure fonts
\setmonofont{Isotype}[Scale=0.8]
\newfontfamily\substitutefont{SimHei}[Scale=0.8]
\setTransitionsForChinese{\begingroup\substitutefont}{\endgroup}

%configure minted to display codes 
\definecolor{Gray}{rgb}{0.9,0.9,0.9}

%remove leading numbers in table of contents
\setcounter{secnumdepth}{0}

%configure section style
%\titleformat{\section}
%    {\normalfont\normalsize}    % The style of the section title
%    {}                    % a prefix
%    {0pt}                % How much space exists between the prefix and the title
%    {\quad}                % How the section is represented
\titleformat{\section}{\large}{}{0pt}{}
\titlespacing{\section}{0pt}{0pt}{0pt}

%enable section to start new page automatically
%\let\stdsection\section
%\renewcommand\section{\penalty-100\vfilneg\stdsection}

%\renewcommand\theFancyVerbLine{\arabic{FancyVerbLine}}
\renewcommand{\theFancyVerbLine}{\sffamily \textcolor[rgb]{0.5,0.5,0.5}{\scriptsize {\arabic{FancyVerbLine}}}}

\setminted[cpp]{
    style=xcode,
    mathescape,
    linenos,
    autogobble,
    baselinestretch=0.9,
    tabsize=2,
    fontsize=\normalsize,
    %bgcolor=Gray,
    frame=single,
    framesep=1mm,
    framerule=0.3pt,
    numbersep=1mm,
    breaklines=true,
    breaksymbolsepleft=2pt,
    %breaksymbolleft=\raisebox{0.8ex}{ \small\reflectbox{\carriagereturn}}, %not moe!
    %breaksymbolright=\small\carriagereturn,
    breakbytoken=false,
}
\setminted[java]{
    style=xcode,
    mathescape,
    linenos,
    autogobble,
    baselinestretch=1.0,
    tabsize=2,
    %bgcolor=Gray,
    frame=single,
    framesep=1mm,
    framerule=0.3pt,
    numbersep=1mm,
    breaklines=true,
    breaksymbolsepleft=2pt,
    %breaksymbolleft=\raisebox{0.8ex}{ \small\reflectbox{\carriagereturn}}, %not moe!
    %breaksymbolright=\small\carriagereturn,
    breakbytoken=false,
}
\setminted[text]{
    style=xcode,
    mathescape,
    linenos,
    autogobble,
    baselinestretch=1.0,
    tabsize=2,
    %bgcolor=Gray,
    frame=single,
    framesep=1mm,
    framerule=0.3pt,
    numbersep=1mm,
    breaklines=true,
    breaksymbolsepleft=2pt,
    %breaksymbolleft=\raisebox{0.8ex}{ \small\reflectbox{\carriagereturn}}, %not moe!
    %breaksymbolright=\small\carriagereturn,
    breakbytoken=false,
}

%configure titles
\title{\LARGE{Beijing Jiaotong University , Standard Code Library} \\
[2ex] \Large{SpadeAce, Fengdalu} }
\date{\today}

%THE SCL BEGINS
\begin{document}
\maketitle

\begin{multicols*}{2}

    \begin{spacing}{0}
        \tableofcontents
    \end{spacing}
\end{multicols*}

\begin{multicols}{2}

\newpage
\begin{spacing}{0.8}

\section{基础}

\subsection{头文件}
\inputminted{cpp}{Basic/headers.cpp}

\subsection{二进制函数 枚举组合数}
\inputminted{cpp}{Basic/枚举组合.cpp}

\subsection{日期时间公式}
\inputminted{cpp}{Basic/zeller.cpp}

\section{IO}

\subsection{Cpp 快速读入}
\inputminted{cpp}{IO/fastio.cpp}

\subsection{Java 模板}
\inputminted{java}{IO/Main.java}

\section{数据结构}

\subsection{BIT}
\inputminted{cpp}{DataStructure/BIT.cpp}

\subsection{LCA}
\inputminted{cpp}{DataStructure/LCA.cpp}

\subsection{RMQ}
\inputminted{cpp}{DataStructure/RMQ.cpp}

\subsection{二维线段树}
\inputminted{cpp}{DataStructure/二维线段树.cpp}

\subsection{主席树}
\inputminted{cpp}{DataStructure/主席树.cpp}

\subsection{树链剖分}
\inputminted{cpp}{DataStructure/树链剖分.cpp}

\subsection{树分治}
\inputminted{cpp}{DataStructure/树分治.cpp}

\subsection{Treap}
\inputminted{cpp}{DataStructure/treap.cpp}

\subsection{Splay}
\inputminted{cpp}{DataStructure/Splay.cpp}

\subsection{KD-Tree}
\inputminted{cpp}{DataStructure/KDT.cpp}

\subsection{KD-Tree 欧几里得距离}
\inputminted{cpp}{DataStructure/KDT2.cpp}


\subsection{DLX 精确覆盖}
\inputminted{cpp}{DataStructure/DLX.cpp}

\subsection{DLX 多重覆盖}
\inputminted{cpp}{DataStructure/DLX2.cpp}

\section{图论}

\subsection{2-SAT}
\inputminted{cpp}{Graph/2-SAT.cpp}

\subsection{KM}
\inputminted{cpp}{Graph/KM.cpp}

\subsection{ISAP}
\inputminted{cpp}{Graph/ISAP.cpp}

\subsection{SAP}
\inputminted{cpp}{Graph/SAP.cpp}

\subsection{Dinic}
\inputminted{cpp}{Graph/dinic.cpp}

\subsection{Dijkstra 费用流}
\inputminted{cpp}{Graph/MinCostFlow.cpp}

\subsection{zkw 费用流}
\inputminted{cpp}{Graph/zkw费用流.cpp}

\subsection{欧拉回路}
\inputminted{cpp}{Graph/欧拉回路.cpp}

\subsection{二分图最大匹配 匈牙利算法}
\inputminted{cpp}{Graph/匈牙利算法.cpp}

\subsection{最小树形图}
\inputminted{cpp}{Graph/朱刘.cpp}

\subsection{哈密尔顿回路}
\inputminted{cpp}{Graph/哈密尔顿回路.cpp}

\subsection{增广路费用流}
\inputminted{cpp}{Graph/增广路费用流.cpp}

\subsection{无向图最小割}
\inputminted{cpp}{Graph/无向图最小割.cpp}

\subsection{一般图最大匹配 带花树}
\inputminted{cpp}{Graph/带花树.cpp}


\subsection{割点/割边}
\inputminted{cpp}{Graph/tarjan.cpp}

\subsection{斯坦纳树}
\inputminted{cpp}{Graph/斯坦纳树.cpp}

\section{字符串}

\subsection{Hash}
\inputminted{cpp}{Strings/BKDRHash.cpp}

\subsection{KMP}
\inputminted{cpp}{Strings/KMP.cpp}

\subsection{EXKMP}
\inputminted{cpp}{Strings/EXKMP.cpp}

\subsection{SA}
\inputminted{cpp}{Strings/SA.cpp}

\subsection{DC3}
\inputminted{cpp}{Strings/DC3.cpp}

\subsection{Manacher 最长回文串}
\inputminted{cpp}{Strings/Manacher.cpp}

\subsection{最小表示法}
\inputminted{cpp}{Strings/最小表示法.cpp}

\subsection{SAM}
\inputminted{cpp}{Strings/SAM.cpp}

\subsection{Trie}
\inputminted{cpp}{Strings/trie.cpp}

\section{数学}

\subsection{Fib}
\inputminted{cpp}{Math/fib.cpp}

\subsection{矩阵}
\inputminted{cpp}{Math/Matrix.cpp}

\subsection{高斯消元}
\inputminted{cpp}{Math/高斯消元.cpp}

\subsection{四边形不等式}
\inputminted{cpp}{Math/四边形不等式.cpp}

\subsection{Java 开根号}
\inputminted{java}{Math/Java开根号.java}

\subsection{Cpp 大数}
\inputminted{cpp}{Math/C++大数.cpp}

\subsection{线性逆元}
\inputminted{cpp}{Math/linear-inversion.cc}

\subsection{勒让德定理}
\inputminted{cpp}{Math/Legendre.cpp}

\subsection{欧拉函数}
\inputminted{cpp}{Math/欧拉函数.cpp}

\subsection{行列式求值}
\inputminted{cpp}{Math/Det.cpp}

\subsection{生成树计数}
\inputminted{cpp}{Math/生成树计数.cpp}

\subsection{Simpson 积分}
\inputminted{cpp}{Math/Simpson.cpp}

牛顿迭代
x1=x0-func(x0)/func1(x0);//进行牛顿迭代计算
我们要求 f(x)=0 的解。func(x)为原方程,func1 为原方程的导数方程
图同构 hash
F t (i) = (F t−1 (i) × A + ∑ F t−1 (j) × B + ∑ F t−1 (j) × C + D × (i == a)) mod P
i→j
j→i
枚举点 a,迭代 K 次后求得的Fk (a)就是 a 点所对应的 hash 值。
其中 K、A、B、C、D、P 为 hash 参数,可自选。


设正整数 n 的质因数分解为 n = ∏pi^ai,则 x^2+y^2=n 有整数解的充要条件是 n 中不存在形
如 pi≡3(mod 4) &(and) 指数 ai 为奇数的质因数 pi

Pick 定理:简单多边形,不自交。(严格在多边形内部的整点数*2 +在边上的整点数– 2)/2 =面积

定理 1:最小覆盖数 = 最大匹配数
定理 2:最大独立集 S 与 最小覆盖集 T 互补。
算法:
1. 做最大匹配,没有匹配的空闲点∈S
2. 如果 u∈S 那么 u 的临点必然属于 T
3. 如果一对匹配的点中有一个属于 T 那么另外一个属于 S
4. 还不能确定的,把左子图的放入 S,右子图放入 T
算法结束

p 是素数且 2^p-1 的是素数,n 不超过 258 的全部梅森素数终于确定!是
n=2,3,5,7,13,17,19,31,61,89,107,127

有上下界网络流,求可行流部分,增广的流量不是实际流量。若要求实际流量应该强算一遍源点出去的流量。
求最小下届网络流:
方法一:加 t-s 的无穷大流,求可行流,然后把边反向后(减去下届网络流),在残留网络中从汇到源做最大流。
方法二:在求可行流的时候,不加从汇到源的无穷大边,得到最大流 X, 加上从汇到源无穷大边后,再求最大流得到 Y。那么 Y 即是答案最小下界网络流。
原因:感觉上是在第一遍已经把内部都消耗光了,第二遍是必须的流量。

平面图一定存在一个度小于等于 5 的点,且可以四染色
( 欧拉公式 ) 设 G 是连通的平面图,n , m, r 分别是其顶点数、边数和面数,n-m+r=2
极大平面图 m≤3n-6

Fibonacci
gcd(2^(a)-1,2^(b)-1)=(2^gcd(a,b))-1.
gcd(F[n],F[m])=F[gcd(n,m)] (牛书,P228)
Fibonacci 质数(和前面所有的 Fibonacci 数互质)
(大多已经是质数了,可能有 BUG 吧,不确定)
定理:如果 a 是 b 的倍数,那么 Fa 是 Fb 的倍数。

二次剩余
p 为奇素数,若(a,p)=1, a 为 p 的二次剩余必要充分条件为 a^((p-1)/2)mod p=1.(否则为 p-1)
p 为奇素数, = a(mod p),a 为 p 的 b 次剩余的必要充分条件为若 a^((p-1)/ (p-1 和 b的最大公约数)) mod p=1.


平方数的和是平方数的问题。
a[0] := 0;
s := 0;
for i := 1 to n - 2 do
begin
	a[i] := a[i - 1] + 1;
	s := s + sqr(a[i]);
end;
{======s + sqr(a[n-1]) + sqr(a[n]) = k^2=======}
a[n - 1] := a[n - 2];
repeat
	a[n - 1] := a[n - 1] + 1;
until odd(s + sqr(a[n - 1])) and (a[n - 1] > 2);
a[n] := (s + sqr(a[n - 1]) - 1) shr 1;
知道 s 和 a[n-1]后,直接求了 a[n].神奇了点。

其实。有当 n 为奇数:n^2 + ((n^2 - 1) div 2)^2 = ((n^2 + 1) div 2)^2
所以有 3 4-- 5 12 -- 7 24 -- 9 40 -- 11 60 ....
a=k*(s^2 - t^2);
b=2*k*s*t
c=k(s^2 + t^2);
则 c^2=a^2+b^2 完全的公式


\end{spacing}
\end{multicols}

\end{document}
